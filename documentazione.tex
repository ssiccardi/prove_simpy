\documentclass{article}
\usepackage[utf8]{inputenc}

\usepackage[colorlinks = true,
            linkcolor = blue,
            urlcolor  = blue,
            citecolor = blue,
            anchorcolor = blue]{hyperref}

\usepackage{minted}
\usemintedstyle{vs}
\usepackage{tabularx}
\usepackage{hyperref}


\title{Documentazione Generatore Tabulati}
\author{Margherita Pindaro }
\date{October 2021}

\begin{document}

\maketitle

\section{Introduzione}
Il generatore è stato realizzato nella fase finale del mio tirocinio, quando il tempo scarseggiava. Per questo motivo ritengo necessarie e suggerisco migliorie al codice, sia in leggibilità, che in testabilità e strutturazione (esempio: creare classe astratta \textsc{Agent}).

Non mi sono limitata ad applicare la documentazione di Simpy ma ho cercato di mettere su un software riutilizzabile, usando un approccio orientato agli oggetti. 

Concetti utili usati da vedere o rivedere: classi con python, design patterns, \textbf{map-filter-reduce}, liste python in generale, operatore condizionale (ternaria), regular expression, tuple, \mintinline{Python}{__init__.py}, list comprehension.

Se hai domande: \href{https://t.me/mpindaro}{@mpindaro} su Telegram.

\section{Simulation}

Questa è la classe eseguibile. Istanza un oggetto della classe \textsc{AgentHandler}. Richiama i suoi metodi di setup (\mintinline{Python}{agentHandler.create_environment()}) e di inizio della simulazione, specificandone la sua durata: 

\mintinline{Python}{agentHandler.start_simulation(env, 15778800)}.

Idealmente in questa classe andrebbero specificati \textbf{tutti} i parametri della simulazione, quindi la durata, le frequenze delle chiamate (o di un altro evento), probabilità etc. Attualmente sono \textit{hardwired} nel codice.

\section{Agent Handler}
La classe più corposa, gestisce tutti gli agenti e la loro interazione. Dato che non avrebbe motivo di esistere più di un agent handler,è un Singleton.

Nell'elenco puntato di seguito verranno spiegato tutti i metodi eccetto i \textit{getter} semplici.


\begin{itemize}
    \item \mintinline{Python}{get_timestamp_last_state_change(self)}. Restituisce il timestamp dell'ultima volta in cui c'è stato un cambiamento di stato nel sistema.
    
    \item \mintinline{Python}{get_timestamp_last_state_change(self)}. Cambia lo stato del sistema con uno nuovo e memorizza il timestamp in cui è avventuo il cambio di stato.
    
    \item \mintinline{Python}{register_log(self, timestamp, event)}. Quando si verifica un evento degno di un log, viene richiamato questo metodo, che lo registra insieme all'istante in cui accade. Metodo di utility.
    
    \item \mintinline{Python}{create_environment([...])}. Inizializza tutti gli agenti e assegna loro un id.
    
    \item \mintinline{Python}{bind(self)}. Associa a ogni agente la lista di altri agenti collegati ad esso. Le relazioni sono simmetriche.
    
    \item \mintinline{Python}{__str__(self)}. Overriding del metodo \mintinline{Python}{str}. Tramite essa è possibile vedere tutte le relazioni tra gli agenti. Metodo di utility.
    
    \item \mintinline{Python}{start_simulation(self, env, duration)}. Metodo che effettivamente esegue la simulazione. Una volta terminata qui vengono salvati su file in memoria persistente il dataset, i log, e gli agenti coinvolti. Sarebbe opportuno fare queste funzionalità in metodi a se stanti.
    
    \item \mintinline{Python}{register_event(self, sender, sender_interc, receiver, receiver_interc,[...])}. Quando si verifica un evento, viene richiamato questo metodo, che registra una riga di dataset. Metodo di utility.
    
    \item \mintinline{Python}{generate_sms_cascade(self, sender, sender_interc, receiver, receiver_interc, timestamp)}. Nel caso l'evento sia un SMS, tramite questo metodo viene generata una conversazione tra i due agenti. Metodo di utility.
    
    \item \mintinline{Python}{handle_call(self, sender, receiver, is_chiamata, duration, timestamp)}. Metodo per gestire una chiamata (o SMS). Nel caso i due agenti siano entrambi non rintracciati l'evento non viene registrato. Se mittente e destinatario sono stessa persona, l'evento non viene registrato. Se l'evento è una chiamata registrerà l'evento altrimenti verrà chiamato \mintinline{Python}{generate_sms_cascade(self, sender,[...])}.
    
    Infine richiama il metodo che che causa ulteriori eventi in base alle specifiche.
    
    \item \mintinline{Python}{innest_events(self, sender, receiver, is_chiamata)}. In seguito a un evento, genera, se necessario, eventi correlati, mandando un interrupt all'agente dovuto. 
    
    \textbf{Importante}: quando viene fatto un interrupt si può specificare una causa (\href{https://simpy.readthedocs.io/en/latest/api_reference/simpy.exceptions.html}{documentazione}). In questo caso ho scelto quindi di formattare le stringhe di causa nel seguente modo: \textsc{<TIPOLOGIA\_EVENTO>\-<CLASSE\_AGENTE><id\_agente\_mittente>}. Quindi, consideriamo un caso in cui un importatore (sender), con id 23, chiama un camionista (receiver). Il camionista in seguito a una chiamata di un importatore chiama uno spacciatore. In questo caso quindi la stringa della causa sarà \textit{chiamata-importatore23}.
    
    \textbf{Questo metodo contiene un errore sulle condizioni, ho messo tutti receiver ma alcune dovrebbero coinvolgere il sender. Da correggere in base alle specifiche.}
    
    \item \mintinline{Python}{get_call_param(self, list_of_receivers)}. Per comodità i parametri delle chiamate (mittente, destinatario, durata etc.) sono creati in questa classe, per un qualsiasi agente. La lista dei destinatari è solitamente una lista di liste.  Metodo di utility.
    
    
\end{itemize}
 
\section{States}
    Enumerativo. Contiene gli stati del sistema.
    
\section{Agenti}

Come detto prima, è necessario la creazione di una classe astratta, \textit{parent} di tutte le classi di agenti. Non spiegherò tutti i metodi di tutte le otto classi, ma solo quelli che trovo significativi. Prenderò d'esempio la classe \textsc{Importatore}, ma i commenti sono generali e valgono per tutte.

\begin{itemize}
    \item \mintinline{Python}{__str__(self)}. Mostra tutti gli agenti con il quale l'agente è collegato.
    
   \item \mintinline{Python}{enter_simulation_environment(self, importatori, esportatori, spacciatori. [...])}. Setup dell'agente, acquisce le sue relazioni e i suoi parametri.
   
    \item \mintinline{Python}{__eq__(self, o: object) -> bool}. Consideriamo due agenti uguali se hanno lo stesso id.
    
    \item \mintinline{Python}{doIKnowPersonX(self, id)}. Metodo che restituisce l'id dell'agente se esso ha, nei propri vettori di relazione, l'agente con id specificato via parametro.
    
    \item \mintinline{Python}{call_someone(self, is_chiamata, duration, receiver)}. Metodo semplice che richiama il metodo \mintinline{Python}{handle_call} della classe AgentHandler.
    
    \item \mintinline{Python}{change_cella(self)}. Aggiorna la posizione corrente dell'agente.
    
    \item \mintinline{Python}{run(self)}. \href{https://simpy.readthedocs.io/en/latest/simpy_intro/process_interaction.html}{Documentazione Simpy}
   
\end{itemize}
  

\end{document}

